\documentclass[a4paper,10pt]{report}
\usepackage[utf8]{inputenc}

% Title Page
\title{Thesis Proposal: \\ From Data to Cognition; Grasping}
\author{Daniel R. Ramírez Rebollo}


\begin{document}
\maketitle

\begin{abstract}
In this document the CODAg (Cognition from Data for Grasping) algorithm is presented briefly. Grasping is a well executed task
for almost any human, we learned since children how to grasp things in several ways, actually a few of the objects that we 
can grasp were thaught to us. We discovered how to grasp things almost naturally, so who could be a better teacher for 
grasping rather than us?
Robots on the other hand, most learn how to grasp several objects depending on their working enviroment that can be
extremely different, and if this wasn't enough, some of them do not have the same hand configuration resulting in a really 
complex problem to solve.

Several approches can be found, from object recognition, RFID tags for objects, shape recognition, among others. We approximate
to the problem with a data driven method, where the teacher would be the human and then let the robot decide which of the graspings
suits better to the object, according to more data from the sensors on the hand itself rather than using image processing. We
support why we did not use image data for the grasping algorithm with Neuroscience discoveries that support a non image data
grasping even in humans.

The Algorithm must be tested in order to provide a measure of its effectiveness, therefore a robotic hand should be built. The
hand by it self is human like, but lacks of sensors and measuring devices. We manage to modify the hand to attach two type of
sensors primaraly, the first type of sensor is a Force Sensitive Resistor, which would provide us the force data of the grasping 
action, the second is a flex sensor wich modifies its value according to the fexion exerted, letting us measure the possible 
angle of each finger. In second plane we added a sensor for acquiring data from actuators and preventing overheat of the motors.

In the first section an introduction to the grasping problem is presented,in chapter two the reader would find the state of 
the art, chapter three will introduce to the reader our data acquisition system, basically how it was build and the whole setup of our grasping 
hand.

\end{abstract}

\chapter{Introduction}

\chapter{State of the Art}

\chapter{Data Acquisition System}

\chapter{Humanoid Hand System}



\end{document}          
